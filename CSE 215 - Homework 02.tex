\documentclass[11pt]{article}
\usepackage[margin=.5in]{geometry}
\usepackage{amsmath}
\usepackage{amssymb}
\setlength{\voffset}{0in}
\setlength{\headsep}{20pt}
\setcounter{section}{+2}
\usepackage{wrapfig,lipsum,booktabs}

\newcommand\tab[1][.8 cm]{\hspace*{#1}}
\newcommand*{\mybox}[1]{\framebox{#1}}

\usepackage{circuitikz}
\usepackage{tikz}
\usetikzlibrary{arrows,shapes.gates.logic.US,shapes.gates.logic.IEC,calc}



\begin{document}
\title{\vspace{-.5in} Ishan Sethi $|$ ID: 110941217 $|$ CSE 215 $|$ Homework 2 in \LaTeX}
\author{Professor McDonnell $|$ Assigned: February 7, 2017 $|$ Due: February 16, 2017}
\date{}
\maketitle
\hrulefill
\tableofcontents

\section{Predicates, Statements, Arguments and Quantified Statements}
\subsection{ - 12, 16b, 16d, 16f, 25d, 26a, 30a, 30c and 33d}
\begin{flushleft}
12) $\forall$ real $x$ and $y$, $\sqrt{x+y} = \sqrt{x} + \sqrt{y}$

Let $x = 10$ and let $y = 10$, this would make $\sqrt{x+y}$ really just be $\sqrt{20}$ which is not equal to $2\sqrt{10}$, coming from $\sqrt{x} + \sqrt{y} = \sqrt{10} + \sqrt{10}$. This would make the latter part of the implied conditional false which would make the ``$\forall$" part false which in the end makes this statement \textbf{false}.

\hrulefill

16 - 

b) $\forall$ real $x$, $x > 0$, $x < 0$ or $x = 0$.

d) $\forall$ logicians $x$, $\sim ($ logicians, $x$, are lazy $)$

f) $\forall$ real $x$, $\sim (x^2= -1)$

\hrulefill

25d) i) $\forall$ irrational $x$, $-x$ is also irrational.

\tab ii) $\forall$ irrational $x$, if $x$ is irrational then $-x$ is also irrational.

\hrulefill

26a) $\forall x$, if $x$ is an integer then it is rational, but $\exists $rational $x$ such that if $x$ is rational then it is not an integer. 

\hrulefill

30a) $\exists x \in \mathbb{Z}$ such that Prime($x$) $\land$ $\sim$Odd($x$)

Let $x = $ 2, 2 is prime and 2 is not odd, this makes the conditional/conjunction true and this makes the whole predicate true because the premises only requires one to satisfy to yield \textbf{true}

\hrulefill

30c) $\exists x \in \mathbb{Z}$ such that odd($x$) $\land$ square($x$)

Let $x=$ 81, 81 is odd and 81 is a perfect square ($9^2 = 81$). This makes the conditional/conjunction true and this makes the whole predicate true because the premises only requires one to satisfy to yield \textbf{true}

\hrulefill

33d) $\mathbb{R} = \{ a, b, c, d\}$ and $(a < b) \land [ (c < d) \Rightarrow (ac < bd) ]$

Let $a = 2$, $b = 3$, $c = -3$ and $d = -2$, this would yield in $(2 < 3) \land [(-3 < -2) \Rightarrow (-6 < -6)]$

$-6 \nless -6$ so this would yield in the conditional being $T \to F$ which outputs a false, which makes the conjunction $T \land F$ which then evaluates to a \textbf{false} predicate.






\hrulefill
\end{flushleft}


\subsection{ - 2, 14, 21, 31, 42, 44}
\begin{flushleft}
 2) Original Statement:  ``All dogs are loyal"
 
 a) ``All dogs are not loyal" (Negation)
 
 b) ``No dogs are loyal" (Negation)
 
 c) ``Some dogs are disloyal" (Negation)
 
 d) ``Some dogs are loyal" (Not a Negation)
 
 e) ``There is a disloyal animal that is not a dog" (Not a Negation)
 
 f) ``There is a dog that is loyal" (Negation)
 
 g) ``No animals that are not dogs are loyal" (Negation)
 
 h) ``Some animals that are not dogs are loyal" (Negation)
 
 
\hrulefill

14) \textbf{Incorrect}

\textbf{Corrected}: $\exists$ real $x_1$ and $x_2$ such that $x_1^2 = x_2^2$ and $x_1 \neq x_2$ 

\hrulefill

21) \textbf{Original}: $\forall$ integers $n$, if $(n$ is divisible by 6$)$, then $(n$ is divisible by 2 and $n$ is divisible by 3$)$.

\textbf{Negation}: $\exists$ integer $n$ such that $(n$ is divisible by 6$)$ and $\sim (n$ is divisible by 2 and $n$ is divisible by 3$)$

\hrulefill

31) \textbf{Original}: $\forall$ integers $n$, if $(n$ is divisible by 6$)$, then $(n$ is divisible by 2 and $n$ is divisible by 3$)$. \textbf{True}

\textbf{Converse}: $\forall$ integers $n$, if $(n$ is divisible by 2 and $n$ is divisible by 3$)$, then $(n$ is divisible by 6$)$. \textbf{True}

\textbf{Inverse}: $\forall$ integers $n$, if $(n$ is not divisible by 6$)$, then $(n$ is not divisible by 2 or $n$ is not divisible by 3$)$. \textbf{True}

\textbf{Contra}: $\forall$ integers $n$, if $(n$ is not divisible by 2 or $n$ is not divisible by 3$)$, then $(n$ is not divisible by 6$)$. \textbf{True}

\hrulefill

42) $\{ \exists$ comprehensive exams $c$ such that if $c$ is not passed, then masters is not obtained.$\}$

\hrulefill

44) $\{ \exists$ person $p$ such that $\sim ($if $p$ has happiness, then $p$ has a high salary. $)\}$

\hrulefill



\end{flushleft}
\subsection{- 10b, 10d, 10f, 11b, 11e, 11f, 19, 24b, 36, 41d and 41g}
\begin{flushleft}

10

b) $\forall$ students $s$, $\exists$  a salad $T$ such that $s$ chose $T$.

This statement is \textbf{False} because Yuen didn't choose a salad.


d) $\exists$ a beverage $b$, such that $\forall$  students  $D$, $D$ chose $b$.

This statement is \textbf{False} because the three people didn't have one drink in common.

f) $\exists$ a station $Z$, $\forall$  students $s$, $\exists$ an item $I$ such that $s$ chose $I$ from $Z$.

This statement is \textbf{True} because all three chose one dish from one station, pie.

\hrulefill

11

b) If you are a student in $S$ and then you have seen Star wars.

e) If student $s$ is in $S$ and student $t$ is in $S$ and $s$ is not the same as $t$, $t$ and $s$ will watch the same movie $m$ in $M$.

f) If student $s$ is in $S$ and student $t$ is in $S$ and $s$ is not the same as $t$, they will watch the same but an undefined movie $m$ in set $M$.

\hrulefill

19) \textbf{Negation:} Any $x$ in ${\rm I\!R}^{+}$ will have one $y$ in ${\rm I\!R}^{+}$ such that $x>y$.

\hrulefill

24b) 
$$\sim (\exists x \in D (\exists y \in E(P(x,y)))) $$
$$\equiv \forall x \in D \sim (\exists y \in E(P(x,y)))$$
$$\equiv \forall x \in D (\forall y \in E(\sim P(x,y)))$$
\hrulefill
\newpage


36) ``Somebody trusts everybody", Let $t(x,y) = $ ``x trusts y''

\textbf{Original: }$\{\exists s \in S, \forall e \in E | t(s,e) \}$

\textbf{Negation: }$\{\forall s \in S, \exists e \in E | \sim t(s,e) \}$

\hrulefill

41d) \textbf{Original: }$\{\forall x \in {\rm I\!R}^{+}, \exists y \in {\rm I\!R}^{+} | xy=1 \}$ 

Yes this is \textbf{True} because if you have a positive real number, the reciprocal is of that is indeed a real number and that times the original = 1 \textit{Q.E.D} (the QED is a joke, if it doesn't apply, dont take points off pls)

g) \textbf{Original: }$\{\forall z\in \mathbb{Z}, \forall y \in \mathbb{Z} | z = x-y \}$

This is \textbf{True} because any real number subtracted from any real number will yield in 1 specific real number. 

\hrulefill

Extra Problem)

``For every object $x$, there is an object $y$ such that $x\neq y$, then $x$ and $y$  have different colors"

Let $dC(x,y) = ``x $ and $y$ have different colors"

a) True
 
b) \textbf{Original: }$\{\forall \text{ objects } x, \exists \text{ object } y, | (x\neq y) \Rightarrow (dC(x,y))\}$

c) \textbf{Negation: }$\{ \exists \text{ object } x, \forall \text{ objects } y, | (x\neq y)\land (\sim dC(x,y)) \} $







\hrulefill
\end{flushleft}
\subsection{- 12, 17, 18, 22, 27, 32}
\begin{flushleft}


12) 

All honest people pay taxes

Darth is not honest

$\therefore$ Darth does not pay taxes

\mybox{This argument is \textbf{invalid} because this is an \textbf{inverse} \textbf{error}.}

\hrulefill

17) 

If an infinite series converges, then its terms go to 0.

The terms of the infinite series $\displaystyle{\sum_{n=1}^{\infty} \frac{1}{n}}$ go to 0.

$\therefore$ The infinite series $\displaystyle{\sum_{n=1}^{\infty} \frac{1}{n}}$ converges.

\mybox{This argument is \textbf{invalid} because this is a \textbf{converse} \textbf{error}.}

\hrulefill

18) 

If an infinite series converges, then its terms go to 0.

The terms of the infinite series $\displaystyle{\sum_{n=1}^{\infty} \frac{n}{n-1}}$ do not go to 0.

$\therefore$ The infinite series $\displaystyle{\sum_{n=1}^{\infty} \frac{n}{n-1}}$ doesn't converge.

\mybox{This argument is \textbf{valid} because \textbf{universal modus tollens.}}

\hrulefill

22) Under the Proof (Number 32)

\hrulefill

27) Under the Proof (Number 32)

\hrulefill
\newpage
Number 32) - Main Proof

${\rm I}$) Let $G(x)$ = ``Grumble from $x$" and $U(x)$ = ``Understand $x$" 

Statement:  $\{ \exists x \in L_e | \sim G(x) \Rightarrow U(x)\}$ 

${\rm I\!I}$) Let $Ar(x)$ = ``$x$ arranged like I am used to" 

Statement:  $\{ \forall x \in A_r | \sim Ar(x)\}$

${\rm I\!I\!I}$) Let $E(x)$ = ``Easy Examples" and $H(x)$ = ``Headache from $x$" 

Statement:  $\{ \forall x \in L_e | E(x) \Rightarrow \sim H(x)\}$ 

${\rm I\!V}$) Phrases already defined

Statement:  $\{ \forall x \in A_r, \exists e \in L_e | \sim Ar(x) \Rightarrow \sim U(e)   \}$

${\rm V}$) Phrases already defined

Statement:  $\{ \exists x \in L_e | \sim H(x) \Rightarrow  \sim G(x) \}$ 

$\therefore$ $\{ \forall  e \in L_e | \sim E(x)      \}$


\begin{center}
	\begin{tabular}{|l|c|} \hline
	
	A) $\{ \forall x \in A_r, \exists e \in L_e | U(e)   \Rightarrow  Ar(x) \}$	& \textbf{Contrapositive of Statement 4} \\ \hline
	%-----------------------------------------------
	A) $\{ \forall x \in A_r, \exists e \in L_e | U(e)   \Rightarrow  Ar(x) \}$	& From Statement A \\ 
	2) $\{ \forall x \in A_r | \sim Ar(x)\}$			& Statement 2 \\ 
   $\therefore \ \{\exists e \in L_e | \sim U(e)\}$& \textbf{B) Universal Modus Tollens} \\ \hline
    %-------------------------------------------------
	5) $\{ \exists x \in L_e | \sim H(x) \Rightarrow  \sim G(x) \}$ 		& Statement 5 \\
	3) $\{ \forall x \in L_e | E(x) \Rightarrow \sim H(x)\}$ 			& Statement 3 \\
	 $\therefore \{ \forall x \in L_e | E(x) \Rightarrow \sim G(x)\} $    & \textbf{C) Universal Transitivity} \\ \hline
	%--------------------------------------------------
	1) $\{ \exists x \in L_e | \sim G(x) \Rightarrow U(x)\}$		& Statement 1 \\
	B) $\{\exists e \in A_r | \sim U(e)\}$			& From Statement B \\
	$\therefore \{\exists e \in A_r | \sim (\sim G(e))\}$		& \textbf{D) Universal Modus Tollens} \\ \hline
	%--------------------------------------------------
	B) $\{\exists e \in A_r | \sim (\sim G(e))\}$				& From Statement D \\
	$\therefore \{\exists e \in A_r | G(e)\}$			& \textbf{E) Universal Double Negation Law} \\ \hline
	%--------------------------------------------------
	C) $\{ \forall x \in L_e | E(x) \Rightarrow \sim G(x)\}$  & From Statement C \\
	E) $\{\exists e \in A_r | G(e)\}$	& From Statement E \\
	$\therefore \{ \forall  e \in L_e | \sim E(x)\}$		& \textbf{ Universal Modus Tollens} \\ \hline
	%--------------------------------------------------
	
	
	\end{tabular}	
	\end{center}





















\hrulefill
\end{flushleft}
\end{document}