\documentclass[11pt]{article}
\usepackage[margin=.5in]{geometry}
\usepackage{amsmath}
\usepackage{amssymb}
\setlength{\voffset}{0in}
\setlength{\headsep}{20pt}
\setcounter{section}{+6}
\usepackage{wrapfig,lipsum,booktabs}
\usepackage{ulem}

\newcommand\tab[1][.8 cm]{\hspace*{#1}}
\newcommand*{\mybox}[1]{\framebox{#1}}


%${\rm I\!R}^3$

\begin{document}
\title{\vspace{-.5in} Ishan Sethi $|$ ID: 110941217 $|$ CSE 215 $|$ Homework 6 in \LaTeX}
\author{Professor McDonnell $|$ Assigned: April 13, 2017 $|$ Due: April 25, 2017}
\date{}
\maketitle
\hrulefill
\tableofcontents

\section{Functions}
\subsection{3b, 3d, 10d, 10e, 10f, 12c, 12d, 41}
\begin{flushleft}

3b) $$g(1) = b\text{, }g(3) = b\text{ and }g(5) =b$$

\vspace{5mm}

3d) $3$ is not the inverse image of $a$, $1$ is the inverse image of $b$.


\hrulefill

10d) $$T(5) = 5\text{ and }1$$

10e) $$T(18) = 1,\  2,\  3,\  6,\  9\text{ and }18$$

10f) $$T(21) = 1,\  3,\  7\text{ and }21$$


\hrulefill

12c) $G(3,\ 2)=((2\cdot(3)+1)mod\ 5,\ (3\cdot  (2)-2)mod\ 5) = \text{\mybox{$ (2,4)$}}$

12d) $G(1,\ 5)=((2\cdot(1)+1)mod\ 5,\ (3\cdot  (5)-2)mod\ 5) = \text{\mybox{$ (3,3)$}}$

\hrulefill

41) $\forall$ subsets of $C$ and $D$ of $Y$, Justify that $F^{-1}(C-D) = F^{-1}(C) - F^{-1}(D)$

\begin{align*}
x\in F^{-1}(C-D)\Rightarrow F(x)\in (C-D) &- \text{Due to the inverse image law} \\
F(x)\in (C-D)\Rightarrow F(x)\in C\text{ and }F(x)\notin D &- \text{Set Difference Law} \\
\intertext{Now proceed to the right side}
x\in F^{-1}(C) - F^{-1}(D) \Rightarrow x\in F^{-1}(C)\text{ and }x\notin F^{-1}(D)  &- \text{Set Difference Law}\\
x\in F^{-1}(C)\text{ and }x\notin F^{-1}(D) \Rightarrow F(x)\in C \text{ and }F(x)\notin D &- \text{Inverse Image Law}\\
\intertext{Since fundamental element argument is the same for both sides, they are both subsets of each other - Definition of Subsets.}
F^{-1}(C-D)\subseteq F^{-1}(C) - F^{-1}(D) &- \text{Definition of Subsets}\\
x\in F^{-1}(C) - F^{-1}(D) \Rightarrow x\in F^{-1}(C)\text{ and }x\notin F^{-1}(D)  &- \text{Set Difference Law}\\
x\in F^{-1}(C)\text{ and }x\notin F^{-1}(D) \Rightarrow F(x)\in C \text{ and }F(x)\notin D &- \text{Inverse Image Law}\\
\intertext{Now proceed to the right side}
x\in F^{-1}(C-D)\Rightarrow F(x)\in (C-D) &- \text{Due to the inverse image law} \\
F(x)\in (C-D)\Rightarrow F(x)\in C\text{ and }F(x)\notin D &- \text{Set Difference Law} \\
\intertext{Since fundamental element argument is the same for both sides, they are both subsets of each other - Definition of Subsets.}
F^{-1}(C) - F^{-1}(D)\subseteq F^{-1}(C-D) &- \text{Definition of Subsets}\\
\intertext{Since they are subsets of each other they are equal $\Rightarrow F^{-1}(C-D)= F^{-1}(C) - F^{-1}(D)$}
\end{align*}

\vspace{-14mm}

\hrulefill 

\end{flushleft}
\subsection{7b, 9c, 9d, 12, 18, 23}
\begin{flushleft}

7b) Since no points map to $g$ in the set of $Y$, the definition of onto is violated and it is \mybox{not onto}.

\hrulefill

9c) If you took all the points from the set $X$ and mapped it to one point out of more than one in the set $Y$, you have a function that is not onto and not one-to-one.

9d) If the set of $X$ are the values $\{ 1,2,3\}$ and the set $Y$ are the values $\{ 1,2,3\}$ and $F: X \to Y$ would result in the mappings $1$ to $2$, $2$ to $3$ and $3$ to $1$, this would be one-to-one, onto and not the identity function.

\hrulefill

12a) $F:\mathbb{Z}\to \mathbb{Z}$ by the rule $F(n) = 2 - 3n$ $\forall n \in \mathbb{Z}$ 

(i) Definition of one-to-one: $\forall x_1$ and $\forall x_2$, if $F(x_1)=F(x_2)$ then $x_1=x_2$

\begin{align*}
F(x_1) &= 2-3(x_1)\\
F(x_2) &= 2-3(x_2)\\
2-3(x_1) &= 2-3(x_2)\\
x_1 &= x_2 
\intertext{This means that the function is one to one.}
\end{align*}

(ii) Definition of onto: $\forall y\in Y$ and $\forall n\in X$, $F(x)=y$.

\begin{align*}
\intertext{Suppose $y\in \mathbb{Z}$}
y &= 2-3(n)\\
n &= \displaystyle\frac{y-2}{3}\\
\text{When }y &= 0\text{, because it can be any integer. }\\
n &= \displaystyle\frac{2}{3} \text{ not an integer.}\\
\intertext{This means that the function is not onto.}
\end{align*}


More on the next page.

\hrulefill

\newpage
12b) $G:\mathbb{R} \to \mathbb{R}$ by the rule $G(x) = 2 - 3x$

Definition of onto: $\forall y \in Y$ and $\forall x \in X$ such that if $F(x)=y$ then the function is onto.

\begin{align*}
\intertext{Suppose $y\in \mathbb{R}$}
y &= 2-3(x)\\
x &= \displaystyle\frac{y-2}{3}\\
F(x) &= 2-3\left(\displaystyle\frac{y-2}{-3}\right) \\
F(x) &= y\\
\intertext{This means that the function is onto as per the definition.}
\end{align*}

\hrulefill

18) $f(x) = \displaystyle\frac{x+1}{x-1}$ $\forall x \in \mathbb{R}$ and $x \neq 1$

Prove if this is one-to-one

Definition of one-to-one: $\forall x_1$ and $\forall x_2$ and $x_1 \neq 1$ and $x_2 \neq 2$, if $F(x_1)=F(x_2)$ then $x_1=x_2$

\begin{align*}
F(x_1) &= \displaystyle\frac{x_1+1}{x_1-1}\\
F(x_2) &= \displaystyle\frac{x_2+1}{x_2-1}\\
\displaystyle\frac{x_2+1}{x_2-1} &= \displaystyle\frac{x_1+1}{x_1-1}\\
x_1x_2 - x_1 +x_2-1 &= x_1x_2 + x_1 - x_2 - 1\\
2x_1 &= 2x_2 \\
x_1 &= x_2 \\   
\intertext{This means that the function is one to one.}
\end{align*}

\hrulefill

23) $H:\mathbb{R}\times \mathbb{R} \to \mathbb{R}\times \mathbb{R}$ by the rule $H(x,y) = (x+1,2-y)$, $\forall (x,y) \in \mathbb{R} \times \mathbb{R}$

\begin{align*}
H(x_1,y_1) &= (x_1+1,2-y_1)\\
H(x_2,y_2) &= (x_2+1,2-y_2)\\
x_1 + 1 &= x_2 + 1 \\
2-y_1 &= 2 - y_2 \\ 
x_1 &= x_2 \\
y_1 &= y_2 \\
\intertext{Thus $(x_1,y_1)=(x_2,y_2)$This is means that it is one-to-one}
\end{align*}





\hrulefill

\newpage

\end{flushleft}
\subsection{7, 10, 17, 20, 24, 25}
\begin{flushleft}

7) $H:\mathbb{Z}\to \mathbb{Z}$ and $K:\mathbb{Z}\to \mathbb{Z}$, $H(a)=6a$ and $K(a)=a\ mod\ 4$ $\forall a \in \mathbb{Z}$

$(K\circ H)(0) = K(H(0)) = K(6(0)) = K(0) = 0$

$(K\circ H)(1) = K(H(1)) = K(6(1)) = K(6) = 2$

$(K\circ H)(2) = K(H(2)) = K(6(2)) = K(12) = 0$

$(K\circ H)(3) = K(H(3)) = K(6(3)) = K(18) = 2$

\hrulefill

10) $G: \mathbb{R}^+ \to \mathbb{R}^+$ and $G^{-1}: \mathbb{R}^+ \to \mathbb{R}^+$, $G(x) = x^2$ and $G^{-1}(x) = \sqrt{x}$ 

$(G\circ G^{-1})(x) = (\sqrt{x})^2 = x = I_x$ 

$(G^{-1}\circ G)(x) = (\sqrt{x^2}) = x = I_x$

We Gucci 

\hrulefill

17) 
\vspace{20mm}

\hrulefill

20) $f:W\to X$ and $g:X\to Y$ and $h:Y\to Z$ must $h\circ (g\circ f) = (h\circ g)\circ f$

Given a $W\in h\circ (g\circ f)$ this essentially means $W\in h(g(f(w)))$

Given a $W\in (h\circ g)\circ f$ this essentially means $W\in h(g(f(w)))$

These are the same making them subsets of each other therefore being equal by the definition of subsets.

\hrulefill

24) $f: \mathbb{R}\to \mathbb{R}$ and $g: \mathbb{R}\to \mathbb{R}$ 

$f(x)=x+3$ $g(x)=-x$

$(g\circ f)=-x-3$

$(g\circ f)^{-1} = -x-3$

$g^{-1} = -x$

$f^{-1} = x-3$

$ (f^{-1} \circ g^{-1}) = -x-3$

They are the same.

\hrulefill

25) $f(x)$ is one-to-one 

\begin{align*}
\intertext{$f(x)$ is one-to-one}
f(x) &= f(y) \\
g(f(x)) &= f(g(y))\text{ Definition of composition}\\
x&=y \text{ by the given condition of $(g\circ f)=I_x$}\\
\intertext{$g(x)$ is one-to-one}
g(x) &= g(y) \\
g(f(x)) &= f(g(y)) \text{ Definition of composition}\\
x&=y \text{ by the given condition of $(g\circ f)=I_y$}\\
\intertext{$f(x)$ is onto}
\text{Suppose } x \in Y \text{ because }(f\circ g) &= I_y = y\\
(f\circ g)(x) &= y\\
f(g(x)) &= y \text{ and }x\in X
\intertext{While $x\in Y$ such that f(g(x)) results in x and $x\in X\ \therefore$ it is ONTO}
\intertext{$g(x)$ is onto}
\text{Suppose }x\in X \text{ because }(g\circ f)(x) &= x\\
(g\circ f)(x) &= x\\
g(f(x)) &= x \text{ and }x\in Y\\
\intertext{While $x\in X$ $g(f(x))$ gives x and $x \in Y \therefore g(x)$ is ONTO} 
\end{align*}

Because $f(g(y)) = y$, both $g^{-1} = f$ and $f^{-1} = g$... the co-domain of one equals the domain of the other.

 





\hrulefill

\end{flushleft}
\end{document}